\documentclass[12pt, a4paper, oneside, scheme=chinese,fontset=yanming]{ctexbook}


% -------------------- 字体宏包 --------------------



% -------------------- 段落宏包 --------------------


% -------------------- 目录宏包 --------------------


% -------------------- 公式宏包 --------------------

\usepackage{amsmath}
\usepackage{mathspec}
\usepackage{amsthm}
% -------------------- 图片宏包 --------------------
\usepackage{float}
\usepackage{graphicx}
\usepackage{subcaption}[labelformat=simple]
\usepackage{caption}[
    font={onehalfspacing, small},
    labelsep=space,              
    skip=6pt,                    
    figurewithin=none,           
    tablewithin=none             
]

% -------------------- 表格宏包 --------------------
\usepackage{array}
\usepackage{booktabs}
\usepackage{multirow}
\usepackage{longtable}
\usepackage{tabularray}
\usepackage{threeparttable}

% -------------------- 列表宏包 --------------------
\usepackage{enumitem}
% -------------------- 交叉宏包 --------------------
\usepackage{cleveref}

% -------------------- 排版宏包 --------------------

% -------------------- 杂项宏包 --------------------
\usepackage{datetime2}


\author{作者}
\title{标题}
\date{\today}


\begin{document}

\chapter{章节标题}

\section{节标题}

\subsection{子节标题}

\subsubsection{子子节标题}

\paragraph{段落标题} 
\subparagraph{子段落标题}
\begin{equation}
    E = mc^2
\end{equation}
% \begin{figure}[h]
%     \centering
%     \includegraphics[width=0.5\textwidth]{example-image-a}
%     \caption{图片标题}
%     \label{fig:example}
% \end{figure}
Some text, and a math formula \(a+b=\sqrt{c}\).
\textbf{\textsf 楷体加粗}
\section{字体示例}
\begin{itemize}
    \item 常规字体:正常文本样式
    \item 罗马体:\textrm{罗马体}
    \item 无衬线体:\textsf{无衬线体}
    \item 等宽字体:\texttt{等宽字体}
    
    \item 字体系列:
    \textmd{中等权重} vs. \textbf{加粗}
    
    \item 字体形状:
    \textup{直立} vs. \textit{斜体} vs. \textsl{伪斜体}
    
    \item 小型大写:\textsc{Small Caps}
    
    \item 强调文本:
    \emph{强调效果(自动切换直立/斜体)}
    
    \item 中文特殊样式:
    {\kaiti 楷书} \quad {\heiti 黑体} \quad {\fsong 仿宋} \quad {\lishu 隶书} \quad {\yahei 雅黑} \quad {\yuanti 圆体} \quad {\songti 宋体}
    
    \item 数学字体:
\end{itemize}

\zihao{3}测试

\end{document}